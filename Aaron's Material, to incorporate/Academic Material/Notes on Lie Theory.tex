\documentclass{article}
\usepackage{amsmath,amssymb,amsfonts}
\usepackage{amsthm}

\newtheorem{theorem}{Theorem}
\newtheorem{corollary}{Corollary}
\newtheorem{lemma}[theorem]{Lemma}
 

 
\begin{document}
	\title{Notes on Lie Theory}
	\author{A.B. Atanasov}
	\date{\today}
	\maketitle
	
	Let $\frak{g}$ be a Lie algebra of dimension $n$. 
	Note for two solvable ideals $\frak{h},\frak{l}$ of $\frak{g}$, their sum $\frak{h}+\frak{l}$ is also solvable:
	$$(\frak{h}+\frak{l})/\frak{h} = \frak{l}/(\frak{h} \cap \frak{l})$$	
	This is the second isomorphism theorem.
	Since $\frak{l}$ is solvable, its quotients will be too, since commutators of this quotient are just quotients of commutators of $\frak{l}$ (this is the fourth isomorphism theorem). 
	
	Assume that a maximal solvable algebra is not unique, so there is another: then take those two and sum them to make a bigger one, condtradicting their maximality. Therefore there is a special maximal solvable subalgebra in $\frak{g}$, and let us call it $\mathrm{rad} \frak{g}$. Now $\frak{g}/\mathrm{rad} \frak{g}$ has no nontrivial solvable ideals, so is semisimple. This means that any Lie algebra can be expressed as an extension of a semisimple Lie algebra by a solvable one. In fact, this extension is a semidirect sum. 
	
	We now prove Engel's Theorem:
	\begin{theorem}[Engel's Theorem]
		If every element in a Lie algebra is nilpotent, then the algebra is nilpotent.
	\end{theorem}
	\begin{proof}
		
	\end{proof}
	
	If every element of $\frak{g}$ is nilpotent then $\frak{g}$ is nilpotent.
	
	If $\frak{g} \subset \frak{gl}_n$ is solvable then there is a basis in which every element is upper triangular.

	The killing form is nondegenerate for a simple Lie algebra (hint, we can assume that it is concrete WLOG)

	Cartan's Criterion: $\frak{g}$ is solvable iff it is orthogonal to $[\frak{g},\frak{g}]$
\end{document}