\documentclass[11pt]{amsbook} 
\usepackage{amsmath,amssymb,amsfonts,amsthm}
\usepackage{enumitem}
\usepackage{graphicx}
\usepackage[noend]{algpseudocode}
\usepackage{algorithm,algorithmicx}
\usepackage{fancyhdr}
\usepackage{geometry}
\usepackage{makeidx}
\geometry{margin = 1.3 in}


\newtheorem{claim}{Claim}
\newtheorem{lemma}{Lemma}
\newtheorem{corollary}{Corollary}

% This will be the book itself, all .tex files compiled into one source. For simplicity, as well as to save space, we're going to have each chapter be a .tex on its own

% Don't forget.. when made into a book odd numbers go on the right, even numbers on the left. This is probably a publisher's issue but it's worth remembering


% When I say "pages" that means two pages = one page front and back

\begin{document} 
	
	Its chill

%Title Page, left

%Back of title page, Right, copyrights and publisher data

%Left, foreword, for around 3-5 pages

%Left, dedication

%Right, blank (or include quote)

%Left, begin ToC, around 2-3 pages

%Left, Acknowledgements

%Right, blank page

%Chapter 0: Philosophy (small chapter)

%Chapter 1: Differential Geometry (large chapter)

%Chapter 2: Representation Theory (large chapter)

%Chapter 3: The algebra SL(2) (medium chapter)

%Chapter 4: Classical Mechanics towards Symplectic Geometry (medium chapter)

%Chapter 5: Applications to General Relativity (large chapter)

%Chapter 6: Classification of Complex Simple Lie Algebras (large chapter, but hopefully not too large)

%Chapter 7: Quantization (medium chapter)

%Right, Notation

%Left, References, hopefully around 3-5 pages of dense text.. don't hold back here

%Index, 

\end{document} 

