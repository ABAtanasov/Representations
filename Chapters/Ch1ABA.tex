\documentclass[../master.tex]{subfiles}
 
\begin{document}

In calculus class you were taught the fundamental theorem, that

	\begin{equation}
		\int_a^b f'(x) dx = f\rvert^b_a
	\end{equation}

And later, in multivariable calculus, you encountered more elaborate integral formulae, such as the divergence theorem of Gauss:

	\begin{equation}
		\int_\Omega \nabla \cdot \mathbf{F} = \int_S \mathbf{F} \cdot dS
	\end{equation}

	where $\Omega$ is the volume of a region we are integrating over, and $S$ is the surface that forms the boundary of $\Omega$. $dS$ then represents an infinitesmal parallelogram that corresponds to a unit area through which $\mathbf{F}$ is flowing out, giving the flux integral on the right. Read in english, Gauss' divergence theorem says ``Summing up the infinitesmal flux over every volume element of the region is the same as calculating the total flux coming out of the region''. The total flux coming out of a region is the sum of its parts over the region.
	
	We will see that both Gauss' theorem and Stokes' theorem are both generalizations of the fundamental theorem of calculus. In fact, they are both two sides of a much more general statement. For the first half of this chapter, we will work towards giving the intuition behind the General Stokes' Theorem\index{Stokes' Theorem!General}.
	
	\begin{equation}
		\int_\Omega d\omega = \int_{\partial \Omega} \omega.
	\end{equation}
	
	On our way, we will begin to slowly move into a much more general setting, beyond the $3$-dimensional world in which most of multivariable calculus was taught. We'll move outside of euclidean spaces, and question what we really mean by the familiar concepts of ``vector'', ``derivative'', and ``distance'' as the bias towards Euclidean geometry no longer remains central in our minds.

\section{The Manifold}

\section{Vector Fields and Forms}

\section{Stokes' Theorem}

\section{Movement, Lie's Ideas}

\section{Distance}

\section{Tensor}

\end{document}