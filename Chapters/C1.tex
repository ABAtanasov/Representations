%!TEX root = /Users/Alex/Work/Representations/Master.tex
\documentclass[../master.tex]{subfiles}

\begin{document}



\chapter{Old Grounds, Revisited}\thispagestyle{empty}
\section{The Cartesian Coordinate System} % (fold)
\label{sec:Cartesian}
	
	One of the most revolutionary ideas in mathematics and physics draws its name from the seventeenth century philosopher Ren\'{e} Descartes. The idea is simple: any point $P$ in the 2D plane can be represented by a pair of real numbers $(x,y)$ in a one-to-one correspondence. The paired numbers represent distances along two perpendicular axes that meet at a special point, $O$, called the \textbf{origin}\index{origin} and correspond to $(0,0)$. This construction gives a \textbf{Cartesian coordinate system}\index{coordinate system!Cartesian} for the Euclidean plane, which does something amazing: it relates the \underline{geometry} of the plane to the \underline{algebra} of variables and equations. Algebra could be represented geometrically, and conversely geometric problems could be solved by transforming them into algebraic ones. The two great classical branches of mathematics, the study of space and the study of quantity, have been joined.
	
	\todofig{Insert 2D Plane with Coordinates}
	
	If the above construction seems obvious to the reader, it should be the same type of obviousness that would strike one when staring at a wheel, perplexed by how such a simple concept went undiscovered for so long in so many cultures. It should speak to the nontrivial character of the Cartesian coordinate system that Descartes himself did not discover the full picture as we know it today. He introduced a variant where distances were measured from a \emph{single} reference axis. It took many years before the idea to add a second perpendicular axis took root, and for the modern treatment to be established.
	
	\todofig{Categorical diagram of the concepts \textbf{POINT} $\leftrightarrow$ \textbf{ASSIGNING SPECIFIC VALUE TO $(x,y)$}, \textbf{CURVE} $\leftrightarrow$ \textbf{EQUATION RELATING VALUES OF $(x, y)$}, \textbf{PLANE} $\leftrightarrow$ \textbf{ALL POSSIBLE VALUES OF $(x, y)$}, \textbf{GEOMETRY} $\leftrightarrow$ \textbf{ALGEBRA}  }
	
	Coordinates soon extended beyond just being pairs of numbers $(x,y)$. Their use applied just as well to describing points in 3D space in reference to a fixed origin by a triple of coordinates $(x,y,z)$. Soon enough after, they were generalized to arbitrarily high dimensions: any point in $n$ dimensional space could be represented by an \textbf{$n$-tuple}\index{vector!as a tuple} $(x_1, x_2, \dots, x_n)$. Despite some controversy at the time of their invention, these tuples of numbers would subsequently be used to lay the foundations for modern physics and mathematics. Modern linear algebra, multivariable calculus, and the myriad of connections between algebra and geometry begin with the concept of a coordinate.
	
	Since then, the use of coordinate systems has proven indispensable to scientists and mathematicians throughout history. Newton used Descartes' ideas to formulate his infinitesimal calculus. Maxwell used it to analyze electromagnetic fields, discovering mathematically that light is a wave in the electromagnetic field. Einstein, going further, made use of coordinate systems on a four dimensional space of three dimensions for physical space and one for time to formulate his theory of relativity and gravitation. Today, physicists and engineers base their calculations in the frames of coordinate systems. In mathematics, Descartes' idea planted the roots for what would turn into the modern field of algebraic geometry. 
	
	When studying a geometry in some $n$-dimensional space, we pick an origin and set of axes to form a coordinate system. Once a Cartesian coordinate system has been set, can represent any point as a tuple of $n$ real numbers. Because of this, $n$-dimensional space is frequently just denoted by $\mathbb{R}^n$, the set of $n$-tuples of real numbers. It is the aim of this chapter to illustrate, in as much clarity as possible, why these two concepts are different. \\
	
	We'll begin this illustration with an example. Let's visualize the simple picture of a ball falling. we could set the origin at some point on the ground, and pick one axis parallel to the ground, ``distance'', and one perpendicular, ``height''. We can decide to measure the axes in meters, or we could decide to measure in feet (nothing stops us from making bad choices of coordinates). The physical point $P$ where the ball lies is represented by $(x,y)=(0\, \mathrm m,10\, \mathrm m)$.
	
	\todofig{Insert Ball Falling}
	
	We can now study motion of the ball, free of geometry, as just two functions $x(t)$ and $y(t)$ on which we can do arithmetic and calculus. If we are given \textbf{equations of motion}\index{equation of motion (EOM)} corresponding to gravitational acceleration downwards: 
	\begin{equation*}
		\frac{d^2y}{dt^2} = -g \, \mathrm{m/s^2}, \hspace{5mm} \frac{d^2x}{dt^2} = 0 \, \mathrm{m/s^2}
	\end{equation*} 
	with initial conditions,
	\begin{equation*}
		x(0) = 0 \,\mathrm m,~ y(0) = 10 \,\mathrm m \hspace{5mm} \frac{dx}{dt}(0) = 10 \, \mathrm{m/s} \hspace{3mm} \frac{dy}{dt}(0) = 0 \mathrm{m/s}
	\end{equation*}
	 then we can perform our well-known kinetic calculations for the system, and see how the system evolves in the time direction\footnote{One might ask ``what is the difference between dynamics and geometry?'', and indeed, when time is introduced as an additional dimension later in the book, we can view dynamics of an $n$ dimensional system as simply a special case of geometry in $n+1$ dimensions: $n$ space and $1$ time.}. In this case we get explicitly:
	 \begin{equation*}
	 	\mathrm x(t) = 10 t \; \mathrm m, \hspace{4mm} y(t) = - \frac{g}{2}t^2 \; \mathrm m
	 \end{equation*}
	 This solution to the equations of motion describes exactly how the coordinates representing the position of the ball change in time. Thus, this can be used to predict the motion of the ball itself. By use of coordinates, this problem of geometric evolution is recast as a calculus problem on real-valued functions of time, and this can be easily solved.
	 
	 But the coordinate system that we chose\textemdash apart from being easy to do calculations in\textemdash was arbitrary. We could have just as well worked in a $45^\degree$ rotated frame, and done the same calculations, with accelerations now being present along the directions of both axes. If we solved this, we would have gotten different-looking equations for the different coordinates $x'(t), y'(t)$, but they would correspond to the exact same \emph{dynamics of motion}.
	
	 \todofig{Two different frames: the original and one rotated 45 clockwise, showing acceleration pointing the same way but different $\ddot x, \ddot y$}
	
	The ball will fall down towards the ground according to the force of gravity. That is the way the world works. It doesn't matter what coordinate system we set up to do that calculation, we should get the exact same result. Plainly, Nature doesn't \emph{care} what coordinate system we use: the events that happen in the natural world are independent of the languages that humans use to describe them. This fact, obvious as it may be, plays tremendous importance in modern physics and will be central to the rest of this book.
	
	As obvious as this fact may be from the physical viewpoint, it gives rise to an extremely strong constraint on the mathematical language that we use to describe the world. \textbf{FINISH HERE}
	
	\todofig{``independence of the physical universe on coordinate representation" Duality diagram physics $\leftrightarrow$ math between something obvious and a profound symmetry principle}
	
	It is funny that coordinate systems, which appear in almost every serious calculation in physics and could be considered the most powerful tool yet developed, must necessarily be so arbitrary. There is no canonical ``God-given'' frame of reference to choose, but to do any calculations, we must nonetheless choose \emph{one}. If this strikes the reader as alien and strange, an esoteric quality reserved to only mathematics, they should turn their mind to the nature of language itself. It is no less arbitrary in its many forms, and no less crucial for developing a deep understanding of the workings of the world. In an even more similar way, the different scales and tunings that form the basis for various forms music around the world are just as arbitrary. No one scale can be demonstrated to be \emph{the} god given one, there are only those that are easier to work with than others.
	
	However, unlike in introductory classes on physics and engineering, where the goal is to become proficient at doing calculations in various coordinate systems to make predictions, this text will take the next steps and look at the ``meta-theory'' behind this. We will ask how the laws of physics can be formulated so that they are independent of the coordinates we use. Doing this will give us significantly greater insight into the nature of physical laws and their underlying \ul{coordinate freedom}.
	Such an understanding will in turn lead to a significantly deeper appreciation of the role that symmetry plays in physics, and ultimately will lead us to understanding more about the physical universe than we ever could before.
	
	% section Descartes (end)
	
	\section{Linear Algebra \& Coordinates} 
	\label{sec:Linear Algebra & Coordinates}% (fold)
	
	\index{linear algebra|(}

	\begin{center}
		\small{\emph{This section assumes some base knowledge of linear algebra. As such, it should be taken as a review and expansion, rather than an introduction.}}
	\end{center}
	
	To begin this refined understanding of the nature of coordinates, let us go to the familiar setting of 3D Euclidean space (read: the physical world as we know it). A Cartesian coordinate system on this space requires a choice of origin, and a choice of axes. The first choice is relatively easy and uninteresting, and comes down to simply marking one point in the space as distinct. The choice of axes, however, is far more interesting, and gives rise to \textbf{linear algebra}\index{linear algebra} and \textbf{bases}\index{basis} on this space.
	
	Given an origin, a choice of axis to measure lengths along consists of picking a \emph{direction} for the axis and a \emph{magnitude} for the length that it measures. You should recognize that these two properties are exactly what define a vector in Euclidean space. To each axis is associated a special vector, $\mathbf e$, corresponding to moving a unit direction along this axis.
	
	\todofig{Draw three axes in 3D with corresponding basis vectors}
	
	Note that we place no requirement that the axes be perpendicular, only that they specify a valid coordinate system. Since we see that specifying an axis is in fact the same as specifying a vector corresponding to moving $+1$ unit along that axis, from now on we will work with vectors. What does it mean for three vectors $\mathbf e_1, \mathbf e_2, \mathbf e_3$ to specify a valid coordinate system in 3D? 
	
	A valid coordinate system for 3D space one that associates to \underline{each} point in 3D space a \underline{unique} set of three real numbers that characterize it. This is the same in any dimension.
	
	In the language of linear algebra, we need our coordinate system's associated vectors to both \underline{\textbf{span}}\index{linear algebra!span} the space so that we can represent any point, and be \underline{\textbf{linearly independent}}\index{linear algebra!linear independence} so that every point will have unique representation in our coordinate system. That's all that a basis is: it specifies a set of axes for a valid coordinate system.
	
	\begin{defn}
		A set of vectors $\{ \mathbf e_1, \dots, \mathbf e_n \}$ is said to span a space if every point $P$ in the space can be represented as a linear combination $a_1 \mathbf e_1 + \dots a_n \mathbf e_n$
	\end{defn}
	
	\begin{defn}
		A set of vectors $\{\mathbf e_1, \dots , \mathbf e_n \}$ is said to be linearly independent if there is only one way to represent the zero vector $\mathbf 0$ as a combination of them, namely as $\mathbf 0 = 0\,\mathbf e_1 + \dots + 0\,\mathbf e_n$.
	\end{defn}
	
	This second definition is the same as saying every vector that we can represent in our system has a unique representation. Let's make this clear. If there were two ways to write a vector $\mathbf v$, as:
	\begin{equation*}
		\begin{aligned}
			a_1 \mathbf e_1 + &\dots + a_n \mathbf e_n = \mathbf v\\
			& \quad \mathrm{and}\\
			b_1 \mathbf e_1 + &\dots + b_n \mathbf e_n = \mathbf v
		\end{aligned}
	\end{equation*}
	then subtracting these two different combinations would give a nonzero way to represent zero, contradicting linear independence:
	\begin{equation*}
		(a_1 - b_1) \mathbf e_1 + \dots + (a_n - b_n) \mathbf e_n = \mathbf 0.
	\end{equation*}
	Conversely, if there were a nonzero combination of vectors summing to zero, then we could add that combination to the coordinate representation of any point and get a \emph{different} representation of the same point, contradicting uniqueness. So \ul{coordinate representations for all vectors are unique as long as there is only one representation for $\mathbf 0$}, namely the one where each component equals zero.
	
	Intuitively, having a set of vector span a space means that this set has enough information to describe every point in the space. Linear independence, on the other hand, means that there is no superfluous information in the set of vectors.  We cannot linearly combine vectors in some subset to get another vector in the set; each vector is adding its own unique additional piece of information, making the set able to span in a new direction. 
	
	Bases that don't span, or are not linearly independent would lead to coordinate systems like these:
	
	\todofig{Show a 2-D basis in a 3-D space, and a basis of 3 vectors in a 2-D subspace}

	Once an origin is chosen, each point in 3D space becomes a vector relative to that origin. Therefore, the idea that points in space are independent of the coordinate system used to describe them applies to vectors just as well. If we choose a basis for our vector space $\mathbf e_1, \dots, \mathbf e_n$, then we can express any vector $\mathbf v$ by a unique combination $\mathbf v = a_1 \mathbf e_1 + \dots + a_n \mathbf e_n$. We then say that in this basis, we can represent $\mathbf v$ by a list of numbers. Often, it is written:
	\begin{equation*}
		\mathbf v = \begin{pmatrix} a_1 \\ \vdots \\a_n	\end{pmatrix}.
	\end{equation*}
	but writing this as an equality is wrong. 
	
	The vector $\mathbf v$ is something physical: a point relative to an fixed origin. As an example, if we chose a stationary frame as an origin, $\mathbf v$ will now represent a physical velocity: the flow of water, the rush of air, the motion of the sun's core relative to Venus. It is purely geometric and has \emph{absolutely nothing to do with numbers} until a coordinate system is chosen. On the other hand, the right hand side by itself is just a list of numbers entirely dependent on the coordinate system. If we change coordinate systems, the right hand side changes. $\mathbf v$ does not.
	
	A geometric vector, $\mathbf v$ is \emph{not} a list of numbers. Once we pick a basis, $u$ can be \emph{represented by} a list of numbers, but if we change into a different basis, those numbers all have to change as well. Later, this exact same idea will be what distinguishes a physical tensor from a multi-dimensional array (like the ones encountered in computer science). It can be \emph{represented by} a multi-dimensional array once a coordinate system is chosen, but the numbers in each entry will differ depending on the coordinate system we pick. 
	
	\begin{dialogue}[Aren't lists of numbers also vectors?]
	Hold on, the definition of vector in linear algebra isn't as something that has magnitude and direction. Vectors are just elements of a vector space: they can be added and scaled by real numbers. You can definitely add and scale lists of $n$ numbers, so aren't they vectors in that sense?
	\end{dialogue}
	
	
	You're completely right, and this is the main source of confusion in understanding this topic. 
	In most courses, we can freely call any list of numbers a `vector'. After all, you can add lists and scale them so they do form a `vector space', as you point out. This boils down to a really unfortunate linguistic degeneracy in mathematics terminology. The type of vectors that we see in physics (acceleration, force, electric field, etc.) are \textbf{geometric vectors}\index{vector!geometric} that have nothing \emph{a priori} to do with lists of numbers until we represent them as such by using coordinate systems. On the other hand, abstract structures that we can add and multiply by scalars are \textbf{algebraic vectors}\index{vector!algebraic}, and lists of numbers are an example of that. Geometric vectors are also an example of algebraic vectors, since they, too, can be added and scaled. 
	
	\todofig{Figure, showing geometric vectors and tuples of numbers as extending the abstract algebraic vectors, and an arrow from tuples to geometric labelled ``representation in a coordinate system''}
	
	To avoid confusing lists of numbers with the geometric vectors, we will refer to the former as \textbf{tuples}\index{vector!as a tuple}.
	
	So, returning to the geometric vector $\mathbf v$, a correct way to write it would be (using standard matrix multiplication convention):
	
	\begin{equation}\label{eq:Repv1}
		\mathbf v = \begin{pmatrix}
			\mathbf e_1  \dots \mathbf e_n
		\end{pmatrix}
		\begin{pmatrix}
			a_1 \\ \vdots \\a_n
		\end{pmatrix} 
		= a_1 \mathbf e_1 + \dots + a_n \mathbf e_n.
	\end{equation}	
	Once we pick a basis, that column of coordinates means something. If we denote our basis $\{\mathbf e_1, \dots, \mathbf e_n \}$ by $B$, then we will use the notation of square brackets to mean not just a column of numbers, but a full combination of the vectors in $B$:
	\begin{equation}\label{eq:Repv2}
		\begin{bmatrix} a_1 \\ \vdots \\a_n	\end{bmatrix}_B := \begin{pmatrix}
			\mathbf e_1  \dots \mathbf e_n
		\end{pmatrix}
		\begin{pmatrix}
			a_1 \\ \vdots \\a_n
		\end{pmatrix} = a_1 \mathbf e_1 + \dots + a_n \mathbf e_n.
	\end{equation}
	
	Now we can begin working towards the goal of understanding coordinate independence of physics by first understanding how coordinates transform as we go from one basis to another. Changes of basis in linear algebra are infamously frustrating to students, because one never knows when to apply a transformation or its inverse. In the following discusion, we will use notation as a tool to make this as painless as possible. 
	
	Let us do a very simple example to start. In the 2-D plane, say we have our original basis $\mathbf e_1, \mathbf e_2$ and we rotate by $\pi/4$ radians to get a new basis. Say we have the vector $\mathbf v = \mathbf e_1 + \mathbf e_2$, i.e. with coordinate representation $\begin{pmatrix} 1 \\ 1 \end{pmatrix}$ in the original basis.
	
	Now our new basis is the old one rotated by $\pi/4$ so 
	\todofig{elementary transform with $1/\sqrt{2}$s}
	\begin{align*}
		\mathbf e_1' &= ~~ \frac{\sqrt 2}{2} \mathbf e_1 + \frac{\sqrt 2}{2} \mathbf e_2\\
		\mathbf e_2' &= - \frac{\sqrt 2}{2}\mathbf e_1 + \frac{\sqrt 2}{2}\mathbf e_2.
	\end{align*}
	Using matrices, this can be written compactly as\footnote{The tuple of basis vectors $\mathbf e_i$ is written as a row rather than a column to be consistent with Equation~\eqref{eq:Repv1}. Then the coordinates are represented in a column. Because of the way we do matrix multiplication, then the matrix must act on the right. It's an issue of styling and indexing, and not physically meaningful. % If we were to write this coordinate transform using columns \& not rows, we'd get a matrix that's the transpose of the one above, and matrix transposes would appear in subsequent equations, making them less tidy.
	}:
	\begin{equation*}
		\begin{pmatrix}
			\mathbf e_1' & \mathbf e_2'
		\end{pmatrix}
		= 
		\begin{pmatrix}
			\mathbf e_1 & \mathbf e_2
		\end{pmatrix}
		\begin{pmatrix}
					 \frac{\sqrt 2}{2} &  -\frac{\sqrt 2}{2} \\
					 \frac{\sqrt 2}{2} &  \frac{\sqrt 2}{2}
		\end{pmatrix}.
	\end{equation*}
	This transformation relates the actual basis vectors themselves. On the other hand, if we wanted to know the \emph{coordinate representations} of the vectors $\mathbf e_1'$ and $\mathbf e_2'$, then in the new basis they would obviously be represented in coordinates as: 
	\begin{equation*}
		\mathbf e_1' = \begin{bmatrix}
			1 \\ 0
		\end{bmatrix}_{\mathrm{new}} \hspace{0.3in}
		\mathbf e_2' = \begin{bmatrix}
			0 \\ 1
		\end{bmatrix}_{\mathrm{new}}
	\end{equation*}
	while in the old basis their representations are:
	
	\begin{equation*}
		\hspace{.3in}\mathbf e_1' = \begin{bmatrix}
			\frac{\sqrt 2}{2} \\ \frac{\sqrt 2}{2}
		\end{bmatrix}_{\mathrm{old}} \hspace{0.3in}
		\mathbf e_2' = \begin{bmatrix}
			-\frac{\sqrt 2}{2} \\ \frac{\sqrt 2}{2}
		\end{bmatrix}_{\mathrm{old}}.
	\end{equation*}
		
	If we know that we can describe a point as \small{$\begin{bmatrix}
					x' \\ y'
		\end{bmatrix}_{\text{new}}$}
	in the new basis, then we can easily get its description in the old basis as:
	\begin{equation*}
		\begin{aligned}
			\begin{bmatrix}
						x' \\ y'
			\end{bmatrix}_{\mathrm{new}} 
			&= x' \mathbf e_1' + y' \mathbf e_2' \\
			&= x' \begin{bmatrix}
				\frac{\sqrt 2}{2} \\ \frac{\sqrt 2}{2}
			\end{bmatrix}_{\mathrm{old}} 
			+ y' \begin{bmatrix}
				-\frac{\sqrt 2}{2} \\ \frac{\sqrt 2}{2}
			\end{bmatrix}_{\mathrm{old}}
			\\ &=\left[ \begin{pmatrix}
				 \frac{\sqrt 2}{2} &  -\frac{\sqrt 2}{2} \\
				 \frac{\sqrt 2}{2} &  \frac{\sqrt 2}{2}
			\end{pmatrix}_{\mathrm{old} \leftarrow \mathrm{new}}
			\begin{pmatrix}
						x' \\ y'
			\end{pmatrix} \right]_{\mathrm{old}} 
		\end{aligned}
	\end{equation*}
	This is the same matrix that related the basis vectors. It's not too hard to see why, and in fact it's just associativity:
	\begin{equation*}
		\begin{aligned}
			\begin{bmatrix}
						x' \\ y'
			\end{bmatrix}_{\mathrm{new}} 
			& = \begin{pmatrix}
				\mathbf e_1' & \mathbf e_2'
			\end{pmatrix} 
			\begin{pmatrix}
				x'\\
				y'
			\end{pmatrix}\\
			 &= \begin{pmatrix}
				\mathbf e_1 & \mathbf e_2
			\end{pmatrix} \begin{pmatrix}
					 \frac{\sqrt 2}{2} &  -\frac{\sqrt 2}{2} \\
					 \frac{\sqrt 2}{2} &  \frac{\sqrt 2}{2}
				\end{pmatrix}
			\begin{pmatrix}
				x'\\
				y'
			\end{pmatrix}\\
			& = \left[ \begin{pmatrix}
					 \frac{\sqrt 2}{2} &  -\frac{\sqrt 2}{2} \\
					 \frac{\sqrt 2}{2} &  \frac{\sqrt 2}{2}
				\end{pmatrix}_{\mathrm{old} \leftarrow \mathrm{new}}
				\begin{pmatrix}
					x'\\
					y'
				\end{pmatrix}
			 \right]_{\mathrm{old}}.
		\end{aligned}
	\end{equation*}
	This also means that given a coordinate tuple $\begin{pmatrix}x\\y\end{pmatrix}$ in the old basis, we can get the coordinates in the new basis by going the \underline{other way} and inverting:
	\begin{equation*}
		\begin{aligned}
			\begin{bmatrix}
				x\\
				y
			\end{bmatrix}_{\mathrm{old}}
			&= \left[ \begin{pmatrix}
					 \frac{\sqrt 2}{2} &  -\frac{\sqrt 2}{2} \\
					 \frac{\sqrt 2}{2} &  \frac{\sqrt 2}{2}
				\end{pmatrix}_{\mathrm{old} \leftarrow \mathrm{new}}
				\begin{pmatrix}
					x'\\
					y'
				\end{pmatrix}
			 \right]_{\mathrm{old}}\\ \Rightarrow 
			 \begin{pmatrix}
			 	x'\\
				y'
			 \end{pmatrix} 
			 &= \begin{pmatrix}
					 \frac{\sqrt 2}{2} &  -\frac{\sqrt 2}{2} \\
					 \frac{\sqrt 2}{2} &  \frac{\sqrt 2}{2}
			\end{pmatrix}^{-1}_{\mathrm{new} \leftarrow \mathrm{old}}
			\begin{pmatrix}
				x\\
				y
			\end{pmatrix}.
		\end{aligned}
	\end{equation*}
	For our vector $\mathbf v$, represented as $(1,1)$ in our original basis, in the new basis, we would have:
	\begin{equation*}
		\mathbf v = \begin{bmatrix}
			1 \\ 1
		\end{bmatrix}_{\mathrm{old}}
		= \left[ \begin{pmatrix}
			 \frac{\sqrt 2}{2} &  -\frac{\sqrt 2}{2} \\
			 \frac{\sqrt 2}{2} &  \frac{\sqrt 2}{2}
		\end{pmatrix}^{-1}_{\mathrm{new} \leftarrow \mathrm{old}} 
		\begin{pmatrix}
			1 \\ 1
		\end{pmatrix}\right]_\mathrm{new}
		= \begin{bmatrix}
			\sqrt 2 \\ 0
		\end{bmatrix}_{\mathrm{new}}.
	\end{equation*}
	Indeed, $\mathbf e_1 + \mathbf e_2 = \sqrt 2 \mathbf e_1' + 0 \mathbf e_2'$.\\
			
	Now, in the general $n$-dimensional case, we start with a basis $\mathbf e_1, \dots, \mathbf e_n$. Now let's make a general transformation into a new basis so that:
	\begin{equation*}
		\begin{aligned}
			\mathbf e_1' &= A_{11} \mathbf e_1 + A_{21} \mathbf e_2 +\dots + A_{n1} \mathbf e_n\\
			\mathbf e_2' &= A_{12} \mathbf e_2 + A_{22} \mathbf e_2 +\dots + A_{n2} \mathbf e_n\\
			& \hspace{1in}\vdots\\
			\mathbf e_n' &= A_{1n} \mathbf e_1 + A_{2n} \mathbf e_2 +\dots + A_{nn} \mathbf e_n,
		\end{aligned}
	\end{equation*}
	or in matrix form:
	\begin{equation}\label{eq:CovariantTransform}
		\begin{aligned}
			\begin{pmatrix}
				\mathbf e'_1 & \dots & \mathbf e'_n
			\end{pmatrix} 
			&= 
			\begin{pmatrix}
				\mathbf e_1 & \dots & \mathbf e_n
			\end{pmatrix}
			\mathbf A_{(\mathrm{old}\rightarrow\mathrm{new})},
		\end{aligned}
	\end{equation}
	\begin{equation*}
		\mathbf A =
		\begin{pmatrix}
			A_{11} & \dots & A_{1n}\\
			\vdots & \ddots & \vdots\\
			A_{n1} & \dots & A_{nn}\\
		\end{pmatrix}.
	\end{equation*}
	
	As before, we see that $\mathbf A$ is the transformation taking the \underline{old} basis $\mathbf e_i$ to the \underline{new} one $\mathbf e'_i$. $\mathbf A$ also takes the \underline{new} coordinate representations and tells us how they'd look like in the \underline{old} basis (note the reversal).
	\begin{equation*}
		\begin{aligned}
			\begin{bmatrix}
				a'_1\\
				\vdots\\
				a'_n
			\end{bmatrix}_{\mathrm{new}}
			&=
			\begin{pmatrix}
				\mathbf e'_1 & \dots & \mathbf e'_n
			\end{pmatrix}
			\begin{pmatrix}
				a'_1\\
				\vdots\\
				a'_n
			\end{pmatrix}\\
			&=
			\begin{pmatrix}
				\mathbf e_1 & \dots & \mathbf e_2
			\end{pmatrix} 
			\mathbf A
			\begin{pmatrix}
				a'_1\\
				\vdots\\
				a'_n
			\end{pmatrix}\\
			&=
			\left[\mathbf A_{(\mathrm{old} \leftarrow \mathrm{new})}
			\begin{pmatrix}
				a'_1\\
				\vdots\\
				a'_n
			\end{pmatrix}
			\right]_{\mathrm{old}}\\
			& = \begin{bmatrix}
				a_1\\
				\vdots\\
				a_n
			\end{bmatrix}_{\mathrm{old}}.
		\end{aligned}
	\end{equation*}
	Note how when $\mathbf A$ acts from the right on the $\mathbf e_i$, it takes the \ul{old basis to the new basis}. On the other hand when it acts on the coordinate tuple, from the left, it took it from \ul{new coordinate old coordinate}. For this reason we can label $\mathbf A$ as $\mathbf A_{(\mathrm{old} \leftrightarrow \mathrm{new})}$ and the side on which it acts (basis vectors vs. coordinate tuples) automatically tells us whether it transforms those quantities from old to new or vice-versa.
	
	So then it is the \emph{inverse} $\mathbf A^{-1}$ that tells us how our old coordinates for a vector $\mathbf v$ will look like in our new basis.
	\begin{equation}\label{eq:ContravariantTransform}
		\begin{pmatrix}
			a_1\\
			\vdots\\
			a_n
		\end{pmatrix} 
		= \mathbf A_{(\mathrm{old} \leftarrow \mathrm{new})} 
		\begin{pmatrix}
			a_1'\\
			\vdots\\
			a_n'
		\end{pmatrix} \Rightarrow 
		\begin{pmatrix}
			a_1'\\
			\vdots\\
			a_n'
		\end{pmatrix} = 
		 \mathbf A^{-1}_{(\mathrm{new} \leftarrow \mathrm{old})}
		\begin{pmatrix}
			a_1\\
			\vdots\\
			a_n
		\end{pmatrix}
	\end{equation}

	From looking at equations~\eqref{eq:CovariantTransform} and~\eqref{eq:ContravariantTransform}, we see the central idea. For any vector $\mathbf v$, if we \emph{vary}\index{covariance} the \underline{basis} $\mathbf e_i$ to a different basis, $\mathbf e_i'$, then the \underline{coordinates} $a_i$ will vary \index{contravariance} the \emph{other} way to $a_i'$, so that the geometric vector
	\begin{equation*}
		\mathbf v = a_1 \mathbf e_1 + \dots + a_n \mathbf e_n = a_1' \mathbf e_1' + \dots + a_n' \mathbf e_n'
	\end{equation*} is \emph{invariant} regardless of coordinate choice:
	\begin{equation*}
		\begin{aligned}
			(\mathbf e_1', \dots, \mathbf e_n') 
			\begin{pmatrix}
				a_1' \\ \vdots \\ a_n'
			\end{pmatrix} 
			= (\mathbf e_1, \dots, \mathbf e_n ) & \mathbf A \mathbf A^{-1} 
			\begin{pmatrix}
				a_1 \\ \vdots \\ a_n
			\end{pmatrix}
			= (\mathbf e_1, \dots, \mathbf e_n ) 
			\begin{pmatrix}
				a_1 \\ \vdots \\ a_n
			\end{pmatrix} = \mathbf v
		\end{aligned}
	\end{equation*}

	We say that the basis vectors $\mathbf e_i$ \textbf{co-vary}\index{vector!covariant} and the coordinates $a_i$ \textbf{contra-vary}\index{vector!contravariant} with the change of basis. The idea, although it sounds simple, is rather hard to get the feel of. It's worth thinking a good bit about how coordinates and bases need to vary in opposite ways so that the physical object represented by the coordinates stays the same regardless of how we look at it. 
	
	\todofig{This will be a caption for a sketch of a 3D (2D?) rotation}
	
	 When you rotate your character in a video game (and in real life too, by the way), the world rotates \emph{contrary} to the direction that you've rotated in. That's because the coordinates of what you see have \emph{contra-varied} while your basis vectors, given by the direction you face have \emph{co-varied}. The end result is that despite changing your coordinate system, physics stays the same: invariant. The universe did not rotate itself just because you did. This extends beyond just rotations to \emph{all} linear transformations. As an example: if you applied a scaling transformation ($\times 100$) to change your units of length: $\mathrm {cm} \to \mathrm m$, the coordinates corresponding to those lengths would change in the opposite way, $100 \to 1$, ($\div 100$) to represent the same distance. \\
	
	\index{linear algebra|)}
	
	\section{The Notion of Length on Vector Spaces} % (fold)
	\label{sec:the_notion_of_length_on_vector_spaces}
	
	Let us consider the property of orthogonality\index{vector!orthogonality}. It's well known that for geometric vectors, there's more that we can do than just add, scale, and transform them: we can take dot products\footnote{And in 3-D we can also take a cross product. This operation, and generalizations, will be discussed in the following chapters.} between them. 
	The dot product is first introduced to in the one-dimensional setting. There, it goes by another name: multiplication. Using the techniques just developed, however, let us look at the one-dimensional picture in a more coordinate-invariant way. First, let us take our physical intuitions about length and translate them into mathematical ideas.
	
	\begin{center}
		\textbf{Physical $\to$ Mathematical:}
	\end{center}
	
	Although the choice of units for measuring distance is always arbitrary, once a unit is chosen, \ul{the shortest distance between two points is a measurable quantity, independent of coordinate system}. Consequently, once an origin is chosen, this allows us to assign a value to each vector $\mathbf v$ in the space, namely its distance from the origin. This is commonly known as the  \textbf{length} or \textbf{norm}\index{vector!norm} of the vector, often denoted as $|\mathbf v|$. Length is a one-dimensional quantity that only depends on measurement along the 1-D space spanned by the vector in question.
	
	On one dimensional Euclidean space---the line---we obtain a vector space upon choosing an origin to call $\mathbf 0$. The choice of origin allows for a way give additive structure to the points on the line: as vectors. Similarly, the choice of length scale allows for multiplication between two vectors $\mathbf a, \mathbf b$. This number is the product of their lengths, with a minus sign if they are anti-parallel. It can be seen as measuring ``the extent to which they agree.''. We call this the \textbf{dot product}\index{dot product}, $\mathbf a \cdot \mathbf b$. Note that \ul{this number cannot be interpreted as a vector on the line}, as (without a basis) there is no god-given negative or positive direction along the line. Note that because of the way the dot product is defined, it is commutative, and scaling one vector correspondingly scales the product: 
	\begin{equation}
		(c \mathbf a) \cdot \mathbf b = c (\mathbf a \cdot \mathbf b) = \mathbf a \cdot (c \mathbf b).
	\end{equation}
	
	If we did have a basis (a single vector, in 1D), $\mathbf e_1$, then such a coordinate system identifies one-dimensional Euclidean space with the \textbf{number line}\index{number line}, $\mathbb R$. The basis vector $\mathbf e_1$ has some length, and we note that the dot product becomes 
	\begin{equation}
		\mathbf a \cdot \mathbf b = (a_1 \mathbf e_1) \cdot (b_1 \mathbf e_1) = a_1 b_1  \, (\mathbf e_1 \cdot \mathbf e_1) = a_1 b_1 |\mathbf e_1|^2
	\end{equation}
	This positive quantity, $|\mathbf e_1|^2$, associated to our basis tells us the relationship between the coordinates we use and the actual length associated with them. In the case that $\mathbf e_1$ is chosen to have unit length, then the coordinates correspond exactly to measurements of length, and the dot product on the line is represented exactly by multiplication on $\mathbb R$. There are two choices of $\mathbf e_1$ that would give such a unit length, and either choice corresponds to picking an orientation for the positive and negative axes.
	
	\todofig{Show the two vectors of unit length, show an example of an anti-parallel dot product}
	
	Can we generalize this one-dimensional multiplication to higher dimensions? We expect it to give a number, not a vector (just like in the 1D case). Let $\mathbf a, \mathbf b$ be two vectors in $n$ dimensional Euclidean space. If we want to define $\mathbf a, \mathbf b$, it suffices to look at just the (two-dimensional) space spanned by them. If we want to reduce this problem back to the one-dimensional setting from before, we can get the component of $\mathbf b$ along the 1-D subspace spanned by $\mathbf a$ by taking an orthogonal projection, as illustrated. Denoting this by $\mathrm{proj}_{\mathbf a} \mathbf b \in \text{span} \{ \mathbf a\}$ we can then take the 1-D product $\mathbf a \cdot \mathrm{proj}_{\mathbf a} \mathbf b$.
	\todofig{Two vectors in the plane, dotted line for orthogonal projection, label theta}
	
	It's easy to see from the above figure that such a product gives
	\begin{equation}
		\mathbf a \cdot \mathbf b = |\mathbf a| |\mathbf b| \cos(\theta)
	\end{equation}
	which is independent of coordinate system. Moreover, its easy to see both geometrically and from the formula above by changing $\theta \to -\theta$ that this product is again symmetric in $\mathbf a$ and $\mathbf b$. It is only slightly harder to see that projection onto $\mathbf a$ is a linear operation, so the dot product is linear in $\mathbf b$. Because of the symmetry property, it is \emph{bilinear} in $\mathbf a$ and $\mathbf b$. To summarize:
	\todofig{Show graphically that projection is linear}
	\begin{prop}[Geometric dot product properties]
		The dot product defined above satisfies the following:
		\begin{enumerate}
			\item Bilinear: ~~ $\begin{matrix}
				(c_1 \mathbf a + c_2 \mathbf b) \cdot \mathbf c = c_1 (\mathbf a \cdot \mathbf c) + c_2 (\mathbf b \cdot \mathbf c)\\
				\mathbf a \cdot (c_1 \mathbf b + c_2 \mathbf c) = c_1 (\mathbf a \cdot \mathbf b) + c_2 (\mathbf a \cdot \mathbf c)
			\end{matrix}$
			\item Symmetric: $\mathbf a \cdot \mathbf b = \mathbf b \cdot \mathbf a$
		\end{enumerate}
	\end{prop}

	This is all independent of any basis. When we use mathematics as a tool for calculations, we first chose a physical basis as our frame of reference, and often chose the directions $\mathbf e_i$ to be perpendicular (so $\mathbf e_i \cdot \mathbf e_j = 0$ for $i \neq j$). In this case we say the basis is \textbf{orthogonal}\index{basis!orthogonal}. Further, we can define the length scale (usually meters) and give each $\mathbf e_i$ a unit length along each axis so that $\mathbf e_i \cdot \mathbf e_i = 1$ for each $i$. We call such vectors \textbf{unit vectors}\index{vector!unit (normalized)} and say they are \textbf{normalized}. If the basis consists of orthogonal vectors that are also unit length, it is called \textbf{orthonormal}\index{basis!orthonormal}. 
	\begin{obs}
		In an orthonormal frame, we have:
		\begin{equation}
			\mathbf a \cdot \mathbf b = (a_1 \mathbf e_1 + \dots + a_n \mathbf e_n) \cdot (b_1 \mathbf e_1 + \dots + b_n \mathbf e_n) = a_1 b_1 + \dots + a_n b_n
		\end{equation}
	\end{obs}
	
	\begin{cor}[Pythagorean Theorem]
		In an orthonormal frame, the length of a vector can be written as
		\begin{equation}
			|\mathbf a| = \sqrt{a_1^2 + \dots + a_n^2}
		\end{equation}
	\end{cor}
	
	When vectors are represented in terms of tuples of numbers, the dot product was taught to us as ``multiply component by component, and then sum that up". This is not, in general, what the dot product really is. Consider a basis transformation as below:
	\begin{align*}
		\mathbf e_1' &= 2 \mathbf e_1 + \mathbf e_2 \\
		\mathbf e_2' &=  \mathbf e_1 + 2 \mathbf e_2.
	\end{align*}
	It's easy to compute the inverse of this matrix and see how the new coordinates should look, but it worthwhile looking at this geometrically. It is not a rotation, but more of a ``stretching''. Notice that while in our original perspective, if we viewed $\mathbf e_1$ and $\mathbf e_2$ as orthogonal vectors, in the \emph{new} perspective, they are \emph{no longer} orthogonal. This is very important: linear transformations in general do not preserve orthogonality. \
	
	\todofig{Graphic of orthogonality being changed under affine transform}
	
	
	
	If we were to take the ``dot product" $\mathbf e_1' \cdot \mathbf e_2'$ just by multiplying corresponding coordinates and summing them up, then in the new basis we'd get zero, but in the original basis we would \emph{not}. It shouldn't come as a surprise, after all
	\begin{equation*}
		\begin{pmatrix}
			a_1\\
			\vdots\\
			a_n
		\end{pmatrix}
		\cdot
		\begin{pmatrix}
			b_1\\
			\vdots\\
			b_n
		\end{pmatrix} =
		\begin{pmatrix}
			a_1 &
			\dots &
			a_n
		\end{pmatrix}
		\begin{pmatrix}
			b_1\\
			\vdots\\
			b_n
		\end{pmatrix}
		=
		a_1 b_1 + \dots + a_n b_n
	\end{equation*}
	is a product of only contravariant coordinates, there is no reason to expect it to produce something invariant. But the state of being perpendicular \emph{is} and invariant in the physical world---independent of any coordinate system. Given what we know from the previous section, we can be more careful, and try to write a dot product in terms of the actual vectors rather than their coordinate-dependent representations,
	\begin{equation*}
		\begin{aligned}
			\mathbf a \cdot \mathbf b
			=& \; (a_1 \mathbf e_1 + \dots + a_n \mathbf e_n) \cdot (b_1 \mathbf e_1 + \dots + a_n \mathbf e_n)\\
			=& \, \quad a_1 b_1 \, (\mathbf e_1 \cdot \mathbf e_1) + \dots + a_1 b_n \, (\mathbf e_1 \cdot \mathbf e_n)\\
			&  \hspace{1.3in} \vdots \\
			& + a_n b_1 \, (\mathbf e_n \cdot \mathbf e_1) + \dots + a_n b_n \, (\mathbf e_n \cdot \mathbf e_n).
		\end{aligned}
	\end{equation*}
	In matrix form, this can be rewritten as
	\begin{equation}\label{eq:DotProductDef}
		\mathbf a \cdot \mathbf b = \begin{pmatrix}
			a_1 &
			\dots &
			a_n
		\end{pmatrix}
		\begin{pmatrix}
			\mathbf e_1 \cdot \mathbf e_1 & \dots & \mathbf e_1 \cdot \mathbf e_n\\
			\vdots & \ddots & \vdots\\
			\mathbf e_n \cdot \mathbf e_1 & \dots & \mathbf e_n \cdot \mathbf e_n
		\end{pmatrix}
		\begin{pmatrix}
			b_1\\
			\vdots\\
			b_n
		\end{pmatrix}.
	\end{equation}
	Because both $\mathbf a$ and $\mathbf b$ in Equation~\eqref{eq:DotProductDef} are geometric vectors, independent of basis, we make the following observation.
	\begin{obs}
		Equation~\eqref{eq:DotProductDef} is independent of the coordinate system used.
	\end{obs}

	Note that Equation~\eqref{eq:DotProductDef} involves just as many products of covariant basis vectors as products of contravariant coordinates. Moreover, every matrix in the above expression is just a matrix of \emph{numbers}. In an arbitrary coordinate frame, this central matrix of the dot products between the basis vectors gives us exactly the information necessary for computing the dot product. It has a special importance, and is called the \textbf{metric}\index{metric tensor}. This matrix is often denoted as $g$, with $g_{ij} := \mathbf e_i \cdot \mathbf e_j$. The use of the lowercase `$g$' is likely in honor of Gauss, who investigated the properties of the metric in differential geometry. 
	 This matrix is symmetric (because the dot product is symmetric), and in the terminology of future chapters it would be called a ``doubly-covariant, rank 2 symmetric tensor''. 
	
	If we know the metric in one basis $\{\mathbf e_i\}$, then we can find it in any other basis $\{\mathbf e_i'\}$ simply by expanding each $\mathbf e_i' \cdot \mathbf e_j'$ in terms of the original basis, and using the old metric:
	\begin{equation*}
		g' = 
		\begin{pmatrix}
			\mathbf e'_1\\
			\vdots\\
			\mathbf e'_n
		\end{pmatrix}
		\begin{pmatrix}
			\mathbf e'_1 & \dots & \mathbf e'_n
		\end{pmatrix} 
		= \mathbf A^T
		\begin{pmatrix}
			\mathbf e_1\\
			\vdots\\
			\mathbf e_n
		\end{pmatrix}
		\begin{pmatrix}
			\mathbf e_1 & \dots & \mathbf e_n
		\end{pmatrix} 
		\mathbf A
		= \mathbf A^T g \mathbf A
	\end{equation*}
	
	We can now use coordinate independence as a beautiful tool to prove the familiar results about the dot product.
	\begin{prop}[The dot product gives lengths]
		The length of a vector is always $|\mathbf a| = \sqrt{\mathbf a \cdot \mathbf a}$
	\end{prop}
	\begin{proof}
		Since we know that both the length and the dot product are independent of the basis used to represent $\mathbf a$, pick a basis with $\mathbf e_1$ the unit vector $\mathbf a/|\mathbf a|$. The rest of the basis is irrelevant, as the remaining coordinates will be zero, since $\mathbf a = |\mathbf a|\, \mathbf e_1$. The dot product of this with itself gives $\mathbf a \cdot \mathbf a = |\mathbf a|^2$.
	\end{proof}
	\begin{cor}[Generalized Pythagorean Theorem]
		\ul{In an orthonormal basis}, the length of a vector is given by
		\begin{equation}
			|\mathbf a| = \sqrt{a_1^2 + \dots + a_n^2}
		\end{equation}
	\end{cor}
	\begin{proof}
		This is immediate from recognizing that the metric is the identity matrix in an orthonormal frame.
	\end{proof}
	\todoex{Proving the pythagorean theorem in $n$D}

	
	The physical world is a Euclidean space, which means we have a way to measure length and angle. When we use mathematics as a tool for calculations, we first chose a physical basis as our frame of reference, and often chose the directions $\mathbf e_i$ to be orthogonal (so $\mathbf e_i \cdot \mathbf e_j = 0$ for $i \neq j$). In this case we say the basis is \textbf{orthogonal}\index{basis!orthogonal}. Further, we can define the length scale (usually meters) and give each $\mathbf e_i$ a unit length along each axis so that $\mathbf e_i \cdot \mathbf e_i = 1$ for each $i$. We call such vectors \textbf{unit vectors}\index{vector!unit (normalized)} and say they are \textbf{normalized}. If the basis consists of orthogonal vectors that are also unit length, it is called \textbf{orthonormal}\index{basis!orthonormal}. In an orthonormal basis, its easy to see that the metric tensor becomes just the identity matrix, and the dot product learned in introductory physics and engineering classes is recovered. 
	
	% \begin{prop}
	% 	In an orthonormal basis $\mathbf e_i$, the length of a vector $\mathbf a$ is given by the square-root of the sum of the squares of its coordinate components,
	% 	\begin{equation}
	% 		|\mathbf a| = \sqrt{a_1^2 + \dots + a_n^2} \hspace{0.5in} \text{(in orthonormal bases).}
	% 	\end{equation}
	% \end{prop}
	% \begin{proof}
	% 	In 1-D this is just the absolute value of $a_1$, and therefore obvious. In 2-D this is exactly the Pythogorean theorem.
	%
	% 	\textbf{DO THIS CLEANLY}
	% \end{proof}
	%
	% Since this is done in an orthonormal basis, it corresponds exactly to taking the square root of the dot product $\mathbf a \cdot \mathbf a$. But this dot product is an \emph{invariant} quantity that has the same value independent of whether the coordinate system chosen is orthonormal (rather than the sum of squares of contravariant coordinates that would only correctly define length in an orthonormal frame). Therefore if the statement that $\mathbf a \cdot \mathbf a$ is the length of $\mathbf a$ holds in one frame, then it must hold in \emph{all} frames, because both quantities are independent of coordinate system.
	% \begin{cor}[The dot product gives lengths]
	% 	The length of a vector is always $|\mathbf a| = \sqrt{\mathbf a \cdot \mathbf a}$.
	% \end{cor}
	
	This type of reasoning is extremely fruitful, and because the dot product (when correctly defined using the metric) gives a value independent of the coordinate system used, we say that it is an \textbf{invariant}\index{invariant}.
	
	\begin{prop}[The dot product gives angles]
		The dot product of two vectors $\mathbf a$ and $\mathbf b$ satisfies the following equality
		\begin{equation}
			\mathbf a \cdot \mathbf b = |\mathbf a| |\mathbf b| \cos(\theta)
		\end{equation}
		where $\theta$ is the angle between the two vectors.
	\end{prop}
	\begin{proof}
		Without loss of generality, we can restrict to the 2D case of the space spanned by $\mathbf a$ and $\mathbf b$.
		In the degenerate case that $\mathbf a$ and $\mathbf b$ are collinear, we write $\mathbf b = c \mathbf a$ and the above proposition follows from the previous corollary. 
		
		In the 2D subspace space given by the span of $\mathbf a$ and $\mathbf b$, we can pick an orthonormal basis $B = \{\mathbf e_1, \mathbf e_2\}$. Further, without loss of generality, we can pick $\mathbf e_1$ to be the unit vector $\mathbf a/|\mathbf a|$. In this basis, the two vectors can be written as
		\begin{equation*}
			\mathbf a = \begin{bmatrix}
				|\mathbf a|\\
				0
			\end{bmatrix}_B
			\quad
			\mathbf b = \begin{bmatrix}
				|\mathbf b| \cos(\theta)\\
				|\mathbf b| \sin(\theta)
			\end{bmatrix}_B.
		\end{equation*}
		This gives us that
		\todofig{Worth putting in a figure here to illustrate 2D subspace basis}
		\begin{equation*}
			\begin{aligned}
				\mathbf a \cdot \mathbf b = |\mathbf a| |\mathbf b| \cos(\theta)
			\end{aligned}
		\end{equation*}
		and because the dot product is an invariant, this must hold in all frames.
	\end{proof}
	
	This is the point of view of going from the physical world into mathematics: we have a set of basis vectors, from which we calculate their several dot products to get the metric, and then we can take dot products of any two arbitrary vectors. However, if we are to understand the physical world by building its description out of mathematical ideas, then we must appreciate that the metric is not something that a vector space has \emph{a priori}. It is something that a vector space must be endowed with.
	
	\begin{center}
		\textbf{Mathematical $\to$ Physical:}
	\end{center}
	
	Consider the space $\mathbf R^n$ of all $n$-tuples. Any choice of basis will give a valid choice of $(1,0, \dots ), (0, 1, \dots), \dots$, but
	
	In a general vector space, there is no one obvious choice of basis. All bases look the same. 
	
	A vector space on its own does not, and so it is called an \emph{affine space}\index{affine space}. In affine space, although there are notions like ``parallel'', there is not a notion of distance. Adding an inner product to affine space gives rise to Euclidean space\index{euclidean space}.
	
	% \todofig{Insert graphic here, with an affine transformation of the plane}
	% \todoadd{Caption something like ``If Euclid's space allowed for not just rotations/translations but also general linear transformations like this, being perpendicular would be a coordinate-dependent idea. Note parallel lines, on the other hand, stay parallel, so affine space \emph{does} have a notion of parallel''}
	
	The takeaway from this discussion is that general linear transformations don't preserve distances. Because of this, the usual definition of the dot product gives different results depending on the choice of coordinate system. To make length an \emph{invariant} regardless of coordinate choice, the definition of the inner product between two vectors needs to be appropriately modified. Without an \emph{inner product} to measure distance, $\mathbb{R}^n$ is \emph{not} Euclidean space $\mathbf{E}^n$, and is called affine space. Affine space looks the same regardless of the linear transformations we apply, while Euclidean space only looks the same when we rotate or translate our frame (because those are the two transformations that don't change lengths). 
	
	The reason we haven't encountered this problem so far in physics is simply because we always have worked in orthonormal frames given by an orthonormal basis like $\hat{\mathbf{i}}, \hat{\mathbf{j}}$, where the inner product in fact \emph{is} just the sum of the products of corresponding components. $\mathbb{R}^n$ without an inner product is affine, not Euclidean. Often however, when talking about Euclidean space, authors refer to it as $\mathbb{R}^n$ instead of $\mathbf{E}^n$ simply because $\mathbf{E}^n$ is modeled as $\mathbb{R}^n$ endowed with an inner product.
	
	
	%
	% In the succeeding chapter, we will study how to define these \emph{inner products}\index{Inner Product} on vectors. To obtain the invariant scalar from the contravariant coordinates, we will need to introduce a doubly-covariant quantity called the \emph{metric tensor}\index{Metric Tensor}.
	%
	%  As soon as we define a product operation between vectors $\mathbf v \cdot \mathbf w$, we have added structure to our vector space. An inner product makes it so that not all bases of vectors are equal. There are bases where the basis vectors satisfy $e_i \cdot e_j = 0$ unless $i = j$, meaning the basis vectors are all orthogonal. The orthogonal bases are special among the set of all bases. Not all linear transformations, therefore, will preserve orthogonal bases: only orthogonal ones will.
	%

	% section the_notion_of_length_on_vector_spaces (end)

	\section[Nonlinear Coordinate Systems are Locally Linear]{Nonlinear Coordinate Systems are\\ Locally Linear}% (fold)
	\label{sec:nonlinear_coordinate_systems_are_locally_linear}
	
	Perhaps you may be wondering why we've spent so much time on changing between coordinate systems represented by basis vectors centered at a fixed origin. Consider the change between Cartesian and polar coordinates. What does this have to do with the linear changes of coordinates that we've been discussing?
	
	We could use something like a polar system of $(r,\theta)$ or a spherical system $(r, \phi, \theta)$. These coordinate systems are not representable in terms of axes, but instead look like this:
	
	\todofig{Graphic of polar coordinate system/spherical} 
	
	This is an example of a non-linear coordinate transformation. They are more commonly referred to as \textbf{curvilinear}. Whereas linear ones map lines to lines, curvilinear ones more generally map lines to curves. The idea for making sure that the equations of physics still stay true for non-linear coordinate transformations is to note that just like a curve locally looks like a line, a \emph{non-linear} transformation locally looks like a \emph{linear} one. The linear transformation that it locally looks like is called the \textbf{Jacobian} \index{Jacobian} $J$. If the laws of physics are invariant under linear transformations locally at each point, then \emph{globally}, they will be invariant under non-linear ones as well. That is why we cared about studying covariance and contravariance for linear transformations: more complicated cases can be reduced to their local linear behavior.
	
	As an example, consider going from a Cartesian to a polar coordinate system. We have $x = r \cos \theta$ and $y = r \sin \theta$. Certainly, this is not a linear transformation of coordinates. There is sinusoidal dependence on $\theta$ in this transformation. Physics and geometry, however, do not have laws in terms of absolute coordinates (it doesn't make sense to say ``That object is located at 50 meters'') but only in terms of relative distances (you'd instead say ``That object is located 50 meters \emph{relative to} me"). It is the changes over relative distances between points that we care about, and these are obtained by integrating the \emph{infinitesimal} changes at each point. 
	
	So although $x,y$ do not depend linearly on theta, through the use of the chain rule, we have a local linear relationship in their infinitesimal changes:
	\begin{align*}
		dx = \cos \theta ~dr - r \sin \theta ~d\theta\\
		dy = \sin \theta ~dr + r \cos \theta ~d\theta
	\end{align*}
	At any given point, this relationship can be written as a linear change of basis.
	\begin{equation*}
		\begin{pmatrix}
			dx \\ dy
		\end{pmatrix}
		 = 
		 \begin{pmatrix}
		 	\cos \theta & - r \sin \theta \\
			\sin \theta & r \cos \theta
		 \end{pmatrix}
		 \begin{pmatrix}
		 	dr \\
			d\theta
		 \end{pmatrix}
	\end{equation*}
	So every nonlinear transformation from some coordinate system $x_1 \dots x_n$ to $x' \dots x_n'$ has the local linear transformation law:
	\begin{align*}
		\begin{pmatrix}
			dx_1 \\ \vdots \\ dx_n 
		\end{pmatrix}_{\text{old}}
		=
	 \begin{pmatrix}
	 	\frac{\partial x_1}{\partial x'_1} & \dots & \frac{\partial x_1}{\partial x'_n} \\
		\vdots & \ddots & \vdots \\
		\frac{\partial x_n}{\partial x'_1} & \dots & \frac{\partial x_n}{\partial x'_n}
	 \end{pmatrix}_{\text{old} \leftarrow \text{new}}
	 \begin{pmatrix}
	 	dx'_1 \\ \dots \\ dx'_n
	 \end{pmatrix}_{\text{new}}\\
	 \Rightarrow 
	 \begin{pmatrix}
	 	dx'_1 \\ \dots \\ dx'_n
	 \end{pmatrix}_{\text{new}}
		=
	 \begin{pmatrix}
	 	\frac{\partial x_1}{\partial x'_1} & \dots & \frac{\partial x_1}{\partial x'_n} \\
		\vdots & \ddots & \vdots \\
		\frac{\partial x_n}{\partial x'_1} & \dots & \frac{\partial x_n}{\partial x'_n}
	 \end{pmatrix}^{-1}_{\text{new} \leftarrow \text{old}}
	\begin{pmatrix}
		dx_1 \\ \vdots \\ dx_n 
	\end{pmatrix}_{\text{old}}
	\end{align*}
	This is exactly the analogue of Equation~\eqref{eq:ContravariantTransform2}, so indeed the changes in coordinates $dx_i$ can be called \emph{contravariant}, just like the coordinates were for the linear transformation case. All of this is just an extension of the principle of local linearity from calculus.
	
	Similarly, we can express the new derivative operators in terms of the old ones by using the chain rule. For polar coordinates we have
	
	\begin{align*}
		\frac{\partial f}{\partial r} = \frac{\partial x}{\partial r}\frac{\partial f}{\partial x}  +  \frac{\partial y}{\partial r} \frac{\partial f}{\partial y} \\
		\frac{\partial f}{\partial \theta} = \frac{\partial x}{\partial \theta}\frac{\partial f}{\partial x}  + \frac{\partial y}{\partial \theta}\frac{\partial f}{\partial y}
	\end{align*}
	or more compactly we can relate just the differential operators themselves: 
	\begin{align*}
		\begin{pmatrix}
			\frac{\partial}{\partial r} \\ \frac{\partial}{\partial \theta}
		\end{pmatrix}
		 = 
		 \begin{pmatrix}
		 	\frac{\partial x}{\partial r} & \frac{\partial y}{\partial r} \\
			\frac{\partial x}{\partial \theta} & \frac{\partial y}{\partial \theta}
		 \end{pmatrix}
		 \begin{pmatrix}
		 	\frac{\partial}{\partial x} \\ \frac{\partial}{\partial y} \\
		 \end{pmatrix}
	\end{align*}
	so that more generally: 
	\begin{equation}\label{eq:partial_transform}
		\begin{pmatrix}
			\frac{\partial}{\partial x'_1} \\ \vdots \\ \frac{\partial}{\partial x'_n} 
		\end{pmatrix}_{\text{new}}
		=
	 \begin{pmatrix}
	 	\frac{\partial x_1}{\partial x'_1} & \dots & \frac{\partial x_1}{\partial x'_n} \\
		\vdots & \ddots & \vdots \\
		\frac{\partial x_n}{\partial x'_1} & \dots & \frac{\partial x_n}{\partial x'_n}
	 \end{pmatrix}_{\text{new} \leftarrow \text{old}}^T
	 \begin{pmatrix}
	 	\frac{\partial}{\partial x_1} \\	\vdots \\ \frac{\partial}{\partial x_n}
	 \end{pmatrix}_{\text{old}}
	\end{equation}
	If you look carefully, you will see that this accordingly mirrors the covariant change of vectors in Equation~\eqref{eq:CovariantTransform2}. So while the infinitesimal changes in the coordinates themselves are contravariant, just like the linear coordinates themselves, the \emph{differential operators} corresponding to changes in these coordinates become \emph{covariant}, just like the vectors $\mathbf e_i$ in the linear case. This is the first correspondence that will hint that our basis vectors $\mathbf e_i$ actually \emph{correspond} to the differential operators $\frac{\partial}{\partial x_i}$.
	
	Indeed, as we begin to move away from 3-D and $n$-D Euclidean space, we will see why the old notions of unit vectors $\mathbf{\hat i}, \mathbf{\hat j}, \mathbf{\hat k}$ are better viewed as the operators $\frac{\partial}{\partial x}, \frac{\partial}{\partial y},$ and $ \frac{\partial}{\partial z}$, and in general $\mathbf e_i \rightarrow \frac{\partial}{\partial x_i}$. This change of notation will allow us to easily pass onto far more general spaces than the Euclidean ones we've gotten used to. 
	
	% section nonlinear_coordinate_systems_are_locally_linear (end)
	
	\section{Einstein's Summation Convention} % (fold)
	\label{sec:einstein's_summation_convention}
	
	Row tuples, column tuples, matrices representing basis transformations (old to new and new to old), co-variance and contra-variance. These ideas have constituted the entirety of this first chapter, and although hopefully they have not been too difficult conceptually, the matrix manipulation even at this early level is already a pain. We have to write everything in terms of tuples, and we need to arbitrarily decide which ones are rows and which ones are columns, and which matrices are transposed so that all the matrix multiplications make sense, as defined in linear algebra. A young physicist named Albert Einstein used a convention\index{summation convention} of writing all these equations so that we did not have to explicitly write out tuples of abstract basis vectors and coordinates.
	
	The first step is to avoid explicitly writing out row tuples and column tuples. To do this, instead of writing out a whole tuple to represent a vector like $(e_1, \dots, e_n)$ we will simply write $e_i$. $e_i$ should be viewed as the whole vector, rather than a specific $ith$ component of $\mathbf v$. The index $i$ is \emph{free} and can be anything. 
	
	The reason it is preferable to view $e_i$ as the whole vector rather than a specific $i$th component of it is because of the same reason given at the end of Section~\ref{sec:Cartesian}, that we never care about just a specific component, but rather the vector as a whole. 
	
	The second step is to be able to differentiate between \emph{covariant} quantities and contravariant quantities. The convention is this: if the quantity is covariant, like a basis vector, then write its index \emph{downstairs}: $\mathbf e_i$. On the other hand, of a quantity is contravariant (like a component of a vector) then write its index \emph{upstairs} as $v^i$ instead\footnote{If you are worried that this will be confused with exponentiation, don't be. In practice, such confusion almost never arises.}. Then we can write Equation~\eqref{eq:Repv1} as 
	
	\begin{equation}
		\mathbf v = \sum_{i} a^i \mathbf e_i
	\end{equation}
	
	For another example, the gradient\index{gradient operator} operator was written $\nabla = (\partial/\partial x_1, \dots, \partial/\partial x_n)$ and is now written simply as $\partial/\partial x^i$ (adopting upper indices for $x^i$ since coordinates contra-vary). This means that Equation~\eqref{eq:partial_transform} can be written as
	\begin{equation}\label{eq:partial_transform2}
		\frac{\partial}{\partial {x'}^i} = \sum_{j} \frac{\partial x^j}{\partial {x'}^i} \frac{\partial}{\partial x^j}.
	\end{equation}
	
	Since $i$ is free to be anything (while $j$ is bound since it is being summed over), this equation holds for every $i$ in $1,\dots, n$ and so is indeed an equation relating two \emph{vectors} on both sides. Note also that this means the transformation matrix can be written as $\frac{\partial x^j}{\partial {x'}^i}$. By writing just a summation and not any explicit matrices, we avoid having to worry about unnecessary troubles like transposes, etc. Note also that this matrix has both upper and lower indices so that the upper index in $\frac{\partial x^j}{\partial {x'}^i}$ gets multiplied with the lower index $\frac{\partial}{\partial x^j}$ to cancel out, and all that remains is the lower index $i$. All the trouble of seeing what's covariant, what's contravariant, and what's invariant is washed away into just seeing how the upper indices and lower indices\index{indices!upper \& lower} cancel out in the end.
	
	Sometimes in the mathematics literature in places where co- and contra-variance are of lesser importance, upper and lower indices are ignored and the notation describing something like a matrix-vector multiplication looks like this:
	\begin{equation*}
		(\mathbf A \mathbf v)_i = \sum_{j} \mathbf A_{ij} \mathbf e_j.
	\end{equation*}
	That is, the $i$th component of the product $\mathbf A \mathbf v$ is the sum over $j$ of the $ij$th component of $\mathbf A$ with the $j$th component of $\mathbf v$. For a matrix-matrix multiplication this would look like
	\begin{equation*}
		(\mathbf A \mathbf{B})_{ik} = \sum_{j} \mathbf A_{ij} \mathbf B_{jk}.
	\end{equation*}
	Notice the pattern: when we wish to multiply these objects, we sum over a common index ($j$ in the above equations) that we make both objects share. The point is \emph{an index is always repeated and summed over} when we do a multiplication. 
	
	Einstein took the bold, but ultimately brilliant step of making the convention that if we ever write an index twice, that \emph{automatically means} that we are summing over it (unless we explicitly say we aren't). The above equations, in Einstein's scheme, now become
	\begin{align*}
		(\mathbf A \vec v)_i & =  \mathbf A_{ij} \vec e_j \\
		(\mathbf A \mathbf{B})_{ik} & = \mathbf A_{ij} \mathbf B_{jk}.
	\end{align*}
	A dot product between $\mathbf v$ and itself would simply be $\mathbf v\cdot \mathbf v = e_i e_i$ in Einstein's convention. Note since $i$ is an index that is summed over (and doesn't appear in the end result), we could just as well replace it by any symbol $e_i e_i = e_j e_j = e_{\ddot \smile} e_{\ddot \smile} = \mathbf v \cdot \mathbf v$.
	
	Now returning back to co-variance and contra-variance, this scheme makes every equation in the chapter shockingly short. Equation~\eqref{eq:Repv1} becomes $\mathbf v = a^i \mathbf e_i$, Equation~\eqref{eq:CovariantTransform2} becomes $\mathbf e_i' = A^j_i \mathbf e_j$, Equation~\eqref{eq:ContravariantTransform2} becomes $a_i' = (A^{-1})^j_i a_j$. Equation~\eqref{eq:partial_transform2} is further reduced to just 
	\begin{equation}\label{eq:covariant_einstein}
		\frac{\partial}{\partial {x'}^i} = \frac{\partial x^j}{\partial {x'}^i} \frac{\partial}{\partial x^j}.
	\end{equation}
	Similarly the transformation of infinitesimal coordinate changes is now just
	\begin{equation}\label{eq:contravariant_einstein}
		dx'^i = \frac{\partial {x'}^i}{\partial x^j} dx^j.
	\end{equation}
	and both of these are really just the chain rule plain and simple.
	
	\todoex{EXERCISE: Check invariance quickly using this method.}

	By doing exercises, this method is very quick and steady to get the hang of. We will adopt it for the rest of the book.

	\todoex{EXERCISE: Show elementwise multiplication of vectors in one basis is a rank (1, 2) tensor in another basis}
	
	\section{Exercises} % (fold)
	\label{sec:exercises1}
	
	% section exercises (end)

\end{document}
